
\documentclass[11pt, A4, oneside]{article}
\usepackage[utf8]{inputenc}

\usepackage{graphicx}
\graphicspath{{./images/}} 

\usepackage[official]{eurosym}

\usepackage{listings}
\usepackage{xcolor}

\definecolor{codegreen}{rgb}{0,0.6,0}
\definecolor{codegray}{rgb}{0.5,0.5,0.5}
\definecolor{codepurple}{rgb}{0.58,0,0.82}
\definecolor{backcolour}{rgb}{0.95,0.95,0.92}

\lstdefinestyle{mystyle}{
	backgroundcolor=\color{backcolour},   
	commentstyle=\color{codegreen},
	keywordstyle=\color{magenta},
	numberstyle=\tiny\color{codegray},
	stringstyle=\color{codepurple},
	basicstyle=\ttfamily\footnotesize,
	breakatwhitespace=false,         
	breaklines=true,                 
	captionpos=b,                    
	keepspaces=true,                 
	numbers=left,                    
	numbersep=5pt,                  
	showspaces=false,                
	showstringspaces=false,
	showtabs=false,                  
	tabsize=2
}

\lstset{style=mystyle}
 
\title{Software for Science, power monitoring \\ The algorithm \\} 

\author{Maurizio Giannattasio, Stefan Schokker, Bram de Wit}

\date{\today} 

\begin{document}

\begin{titlepage}
	\maketitle
	
	\begin{figure}[htbp]
		\centering
		\includegraphics{PowerMonitoring_logo}
	\end{figure}	
	\begin{figure}[htbp]
		\centering
		\includegraphics{Logo-astron}
	\end{figure}

\end{titlepage}


\begin{abstract}
	
In this paper we will explore what algorithm we should pick to test on our platform. The algorithm should be relevant to radio astronomy and provide an interesting study case. 
	
\end{abstract}

\tableofcontents

\newpage

\section{Algorithms in radio astronomy}

At first we wanted to know what algorithms were relevant to our project, since we are doing this project for Astron\cite{astron}, the algorithm we are going to use should have some relevance in radio astronomy. After some searching we came up with a few algorithms that are used in that field.
\begin{itemize}
	\item Machine learning (Artificial intelligence)
	\item Auto tuning
	\item CLEAN (image deconvolution)
	\item FFT (Fast Fourier Transform)
\end{itemize}

\subsection{Machine learning}

Machine learning can be used in a variety of ways, and thus it also has possible applications in radio astronomy. In a paper by Cornelis Johannes Wolfaardt\cite{Machine_learning} the possibility to use machine learning to automatically classify cases of radio frequency interference is explored. \par 
Or in another paper by E.M. Howard\cite{Data_analysing} the problem of large amounts of data that needs processing is tackled. Datasets being produced by new telescopes like the SKA\footnote{Square Kilometre Array} are reaching terabytes of data and in the future will get up to petabytes. Analysing all that data would be a huge task for humans, machine learning might be a solution there. 

\subsection{Auto tuning}

We found a rather interesting paper that explores using autotuning and many core accelerators to improve efficiency of other algorithms\cite{Autotune}. Autotuning is in essence a technique to set your parameters to their best possible configuration. And therefore can be used to set the parameters of an algorithm. In this paper they find that using autotune in combination with many core accelerators can lead to a significant boost in performance.

\newpage

\subsection{CLEAN}

The clean algorithm is an image deconvolution algorithm first described by J.A. Högbom in 1974.\cite{CLEAN} Images in radio astronomy are often spread out a bit making single points look like blurs. The clean algorithm fixes this by taking the brightest point and subtracting a small portion of that brightness over the entire image. It does this repeatedly until the image looks "clean".(simplified explanation)

\subsection{Fast Fourier Transform}

The Fast Fourier Transform(FFT) is a mathematical transformation to a signal showing the frequencies most present in a certain signal. It it used for e.g. pulsar detection, image processing, analysing the frequency spectrum and more. The FFT is one of the most fundamental algorithms for data processing in radio astronomy. 

\section{Our choice}

We have decided to go for the Fast Fourier Transformation, since most of the other algorithms use images that are usually already processed by an FFT. The CLEAN algorithm uses FFT images, and the machine learning in turn uses cleaned images. The FFT will provide an interesting study case, as it is relevant in radio astronomy, and still feasible to set up on our platform within the limited duration of this project.    

\bibliographystyle{ieeetr}
\bibliography{bibliography}

\end{document}